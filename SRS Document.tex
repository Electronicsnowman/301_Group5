\documentclass[10pt,a4paper]{article}
\usepackage[utf8]{inputenc}
\usepackage{amsmath}
\usepackage{amsfonts}
\usepackage{amssymb}
\author{Group 5}
\title{Mini project SRS}

\begin{document}
\tableofcontents
\pagebreak
\section{Introduction}
\subsection{Purpose}
The purpose of this document is to form a legally binding common ground between the stakeholders and the developers. This document will stipulate the functional as well as non-functional requirements of a system that will allow various contributors to collect, navigate and manipulate the marks of students studying at the university of Pretoria, as well as allow the students to view their marks.
\subsection{Document conventions}
\begin{description}
\item Documentation formulation: LaTeX
\item Entity Relationship Diagram Notation: Crows Foot 
\end{description}
\subsection{Project cope}
This project ,when finalized, should act as a holistic marks management system. The system will have different privilege levels, when assigned to a level , users will have a restricted set of tools stipulated as follows: 
\begin{enumerate}
\item Level 1 users (such as Students) should be able to view their own marks.
\item Level 2 users (such as Teaching Assistants and Tutors) will be able to view any students marks that are in a session that the level 2 person is assigned to. This level will also give access to adding as well as modifying marks.
\item Level 3 users (such as lecturers) will be able to configure mark roll-ups, add assessment opportunities and draw reports from all of the marks of the subjects of which he is assigned to. This level will also have all of the privileges of level 1 as well as level 2 users.
\item Level 4. This level can be seen as a level with "root" privileges. It is intended to only be assigned to either one or two people and will give access to be able to all level 1, 2 ,3 and 4 privileges as well as be able to add and remove subjects from the system. 
\end{enumerate}
These tools will be available in a web based interface as well as an Android interface.
\subsection{References}

Jan kroeze : by means of requirement extraction at meetings on 7 and 14 February 2014

\section{System description}
description text here
\section{Functional requirements}
	\subsection{System features}
	\section*{Mark API}
	\begin{itemize}
		\item Must interface with CS website login details or is it LDAP?
		\item Must use Soap interface through out all applications.
		\item Must Facilitate marking sessions.
		\item Marking sessions must coincide with practical sessions.
		\item Marking session durations must also be personalizable by the client.
		\item Marker(s) must be able to record marks during marking sessions.
		\item Marks must be stored in a database.
		\item Database must be exportable to a .csv file.
		\item Database must have lock/unlock capabilities that are also automated depending on whether are marking session is currently happening.
		\item Unlocked databases must allow marker(s) to alter marks.
		\item Altered marks must include reason for alteration.
		\item Marker(s) must be able to alter marks from previous marking session only with the permission/knowledge of the client.
	\end{itemize}
	\subsection*{Android User Interface}
	\begin{itemize}
		\item Marker(s) must be able to select a marking session(s).
		\item Student(s) must be able to check acquired marks.
		\item Marks recorded by marker must be stored on local database.
		\item Local database must synchronize with server database.
	\end{itemize}
	\subsection*{Web User Interface}
	\begin{itemize}
		\item Must be implemented through Django.
  		\item Students must be able to view their marks
  		\item Must facilitate markers/students who don't use android enabled mobile devices.
	\end{itemize}
	\section*{Booking API}
	Must interface with Computer Science website booking system.
\subsection{Use-Cases}
\section{External interface requirements}
Add subsections as needed
\section{Technical/Non-functional requirements}
Add subsections as needed
\section{Open issues}
Add open issues here
\section{Glossary}
Add glossary text here
\end{document}