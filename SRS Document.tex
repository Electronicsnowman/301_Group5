\documentclass[10pt,a4paper]{article}
\usepackage[utf8]{inputenc}
\usepackage{amsmath}
\usepackage{amsfonts}
\usepackage{amssymb}
\author{Group 5}
\title{Mini project SRS}

\begin{document}
\tableofcontents
\pagebreak
\section{Introduction}

\subsection{Purpose}
The purpose of this document is to form a legally binding common ground between the stakeholders and the developers. This document will stipulate the functional as well as non-functional requirements of a system that will allow various contributors to collect, navigate and manipulate the marks of students studying at the university of Pretoria, as well as allow the students to view their marks.

\subsection{Document conventions}
\begin{description}
\item Documentation formulation: LaTeX
\item Entity Relationship Diagram Notation: Crows Foot 
\end{description}

\subsection{Project cope}
This project ,when finalized, should act as a holistic marks management system. The system will have different privilege levels, when assigned to a level , users will have a restricted set of tools stipulated as follows: 
\begin{enumerate}
\item Level 1 users (such as Students) should be able to view their own marks.
\item Level 2 users (such as Teaching Assistants and Tutors) will be able to view any students marks that are in a session that the level 2 person is assigned to. This level will also give access to adding as well as modifying marks.
\item Level 3 users (such as lecturers) will be able to configure mark roll-ups, add assessment opportunities and draw reports from all of the marks of the subjects of which he is assigned to. This level will also have all of the privileges of level 1 as well as level 2 users.
\item Level 4. This level can be seen as a level with "root" privileges. It is intended to only be assigned to either one or two people and will give access to be able to all level 1, 2 ,3 and 4 privileges as well as be able to add and remove subjects from the system. 
\end{enumerate}
These tools will be available in a web based interface as well as an Android interface.

\section{Vision}
The vision of the project is to create multiple interfaces that communicate with a centralized database to record mark results, audit, manage, manipulate, modify and report results based on these results into a pdf or csv format.
The system will be vastly scalable and allow multiple users to interface with the application at the same time. It will provide resource management utilities that will cater to the performance and reliability. Thus the client’s actions will be correctly reflected on the system.

\section{Background}
Lecturers, Tutors and Teaching assistants often find that recording and evaluating assessments that are issued to students is very inconvenient and inefficiently done. Even more so if a recoding or multiple need to be modified.
It was also troublesome to generate reports for different platforms as well as perform statistical analysis of the data to determine vital constraints such as the performance of the students in assignments. And use the data to better improve on the already implemented schema.
So with this in mind, a centralized schema is designed where multiple clients can perform data input, output, manipulations and reporting at the same time, and to do so on different interface such as on conventional computers and mobile phones running android as well as on laptops and tablets.
These clients with communicate over a protected (encrypted) and synchronized framework to enable efficient and secure transactions. 


\section{Architecture requirements}
\subsection{Access channel requirements}
\subsection{Quality requirements}
\subsection{Integration requirements}
\subsection{Architecture constraint}

\section{Functional Requirements}
\subsection{Introduction}
\subsection{Scope and Limitations}
\subsection{Required functionality}
\subsection{Use case prioritization}
\subsection{Use case/Services contracts}
\subsection{Process specifications}
\subsection{Domain Objects}

\section{Open Issues}
\section{Glossary}
\end{document}
