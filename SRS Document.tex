\documentclass[10pt,a4paper]{article}
\usepackage[utf8]{inputenc}
\usepackage{amsmath}
\usepackage{amsfonts}
\usepackage{amssymb}
\author{Group 5}
\title{Mini project SRS}

\begin{document}
\tableofcontents
\pagebreak
\section{Introduction}
intro text here
\subsection{Purpose}
purpose text here
\subsection{Document conventions}
convention text here
\subsection{Project cope}
scope text here
\subsection{References}
reference text here
\section{System description}
description text here
\section{Functional requirements}
	\subsection{System features}
	\section*{Mark API}
	\begin{itemize}
		\item Must interface with CS website login details.
		\item Must use Soap interface through out all applications.
		\item Must Facilitate marking sessions.
		\item Marking sessions must coincide with practical sessions.
		\item Marking session durations must also be personalizable by the client.
		\item Marker(s) must be able to record marks during marking sessions.
		\item Marks must be stored in a database.
		\item Database must be exportable to a .csv file.
		\item Database must have lock/unlock capabilities that are also automated depending on whether are marking session is currently happening.
		\item Unlocked databases must allow marker(s) to alter marks.
		\item Altered marks must include reason for alteration.
		\item Marker(s) must be able to alter marks from previous marking session only with the permission/knowledge of the client.
	\end{itemize}
	\subsection*{Android User Interface}
	\begin{itemize}
		\item Marker(s) must be able to select a marking session(s).
		\item Student(s) must be able to check acquired marks.
		\item Marks recorded by marker must be stored on local database.
		\item Local database must synchronize with server database.
	
		\item \subsubsection*{Participant Login}
		\begin{itemize}
			\item Participants must enter their username and password.
			\item Authorization will determine if participant is a student or marker. 
			\item Eligible students should be directly shown the view marks screen.
			\item Eligible markers should be given an option to choose from the Record Marks function to the View Marks function.
		\end{itemize}
		
		\item \subsubsection*{View Marks API}
		\begin{itemize}
			\item Participants must be shown a selection of all the modules they are enrolled for which use this system. 
			\item Once a module has been selected by a student the participant acquired marks are shown to them.
		\end{itemize}
		
		\item \subsubsection*{Record Marks API}
		\begin{itemize}
			\item Marker is shown all the marking sessions he/she is enlisted for.
			\item Once marker selects a marking session they can search a specific student by name/student number to which marks will be recorded for.
			\item Altering marks works in a similar fashion.
		\end{itemize}
	\end{itemize}
	\subsection*{Web User Interface}
	\begin{itemize}
		\item Must be implemented through Django.
  		\item Students must be able to view their marks
  		\item Must facilitate markers/students who don't use android enabled mobile devices.
  		
		\item \subsubsection*{Participant Login}
		\begin{itemize}
			\item Participants must enter their username and password.
			\item Authorization through LDAP will determine if participant is a student or marker. 
			\item Eligible students should be directly shown the view marks screen.
			\item Eligible markers should be given an option to choose from the Record Marks function to the View Marks function.
		\end{itemize}
		
		\item \subsubsection*{View Marks API}
		\begin{itemize}
			\item Participants must be shown a selection of all the modules they are enrolled for which use this system. 
			\item Once a module has been selected by a student the participant acquired marks are shown to them.
			\item Provide an export marks function for both .csv and .pdf files.
		\end{itemize}
		
		\item \subsubsection*{Record Marks API}
		\begin{itemize}
			\item Marker is shown all the marking sessions he/she is enlisted for.
			\item Once marker selects a marking session they can search a specific student by name/student number to which marks will be recorded for.
			\item Altering marks works in a similar fashion.
		\end{itemize}
	\end{itemize}
	\section*{Booking API}
	Must interface with Computer Science website booking system.
\subsection{Use-Cases}
\section{External interface requirements}
Add subsections as needed
\section{Technical/Non-functional requirements}
Add subsections as needed
\section{Open issues}
Add open issues here
\section{Glossary}
Add glossary text here
\end{document}