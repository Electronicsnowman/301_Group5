\documentclass[10pt,a4paper]{article}
\usepackage[utf8]{inputenc}
\usepackage{amsmath}
\usepackage{amsfonts}
\usepackage{amssymb}
\author{Group 5}
\title{Mini project SRS}

\begin{document}
\tableofcontents
\pagebreak
\section{Introduction}
\subsection{Purpose}
The purpose of this document is to form a legally binding common ground between the stakeholders and the developers. This document will stipulate the functional as well as non-functional requirements of a system that will allow various contributors to collect, navigate and manipulate the marks of students studying at the university of Pretoria, as well as allow the students to view their marks.
\subsection{Document conventions}
\begin{description}
\item Documentation formulation: LaTeX
\item Entity Relationship Diagram Notation: Crows Foot 
\end{description}
\subsection{Project cope}
This project ,when finalized, should act as a holistic marks management system. The system will have different privilege levels, when assigned to a level , users will have a restricted set of tools stipulated as follows: 
\begin{enumerate}
\item Level 1 users (such as Students) should be able to view their own marks.
\item Level 2 users (such as Teaching Assistants and Tutors) will be able to view any students marks that are in a session that the level 2 person is assigned to. This level will also give access to adding as well as modifying marks.
\item Level 3 users (such as lecturers) will be able to configure mark roll-ups, add assessment opportunities and draw reports from all of the marks of the subjects of which he is assigned to. This level will also have all of the privileges of level 1 as well as level 2 users.
\item Level 4. This level can be seen as a level with "root" privileges. It is intended to only be assigned to either one or two people and will give access to be able to all level 1, 2 ,3 and 4 privileges as well as be able to add and remove subjects from the system. 
\end{enumerate}
These tools will be available in a web based interface as well as an Android interface.
\subsection{References}

Jan kroeze : by means of requirement extraction at meetings on 7 and 14 February 2014

\section{System description}
description text here
\section{Functional requirements}
	\subsection{System features}
	\section*{Mark API}
	\begin{itemize}
		\item Must interface with CS website login details or is it LDAP?
		\item Must use Soap interface through out all applications.
		\item Must Facilitate marking sessions.
		\item Marking sessions must coincide with practical sessions.
		\item Marking session durations must also be personalizable by the client.
		\item Marker(s) must be able to record marks during marking sessions.
		\item Marks must be stored in a database.
		\item Database must be exportable to a .csv file.
		\item Database must have lock/unlock capabilities that are also automated depending on whether are marking session is currently happening.
		\item Unlocked databases must allow marker(s) to alter marks.
		\item Altered marks must include reason for alteration.
		\item Marker(s) must be able to alter marks from previous marking session only with the permission/knowledge of the client.
	\end{itemize}
	\subsection*{Android User Interface}
	\begin{itemize}
		\item Marker(s) must be able to select a marking session(s).
		\item Student(s) must be able to check acquired marks.
		\item Marks recorded by marker must be stored on local database.
		\item Local database must synchronize with server database.
	\end{itemize}
	\subsection*{Web User Interface}
	\begin{itemize}
		\item Must be implemented through Django.
  		\item Students must be able to view their marks
  		\item Must facilitate markers/students who don't use android enabled mobile devices.
	\end{itemize}
	\section*{Booking API}
	Must interface with Computer Science website booking system.
\subsection{Use-Cases}
\section{External interface requirements}
\section{External interface Requirements}
\label{External interface Requirements}


\subsection{User Interfaces}
\label{User Interfaces}


\subsubsection{Login Interface}
\label{Login Interface}
\begin{itemize}
\item Users should be prompt to provide their student/personnel number and password on their interaction with the system.	
\item The system will notify the user if the login information is invalid or navigate to the user's home page if the information is valid.
\end{itemize}
\subsubsection{Student interface}
\label{Student interface}
\begin{itemize}
\item Students should be able to view a list of the subjects they are enrolled for. Each subject should link to all its activity marks page.
\item Students should be able to export all their marks for a specific subject or all subjects.
\end{itemize}
\subsubsection{Teaching Assistance or Marker interface}
\label{Teaching Assistance or Marker interface}

\begin{itemize}
\item Markers should be able to view the list of all the subjects they are responsible for.
\item Markers should also be able to view all the sessions they are responsible for, along with their venues and times.
\item Markers should also be able to view a list of students booked for that session.
\item Marker should be able to view previous and current session marks, and they should be able to update those marks.
\item Marker should be able to search for a student by providing name, surname, or student number.
\end{itemize}

\subsubsection{Lecture interface}
\label{Lecture interface}

\begin{itemize}
\item The lecture should be able to view the list of all courses they are responsible for. Each course should link to an evaluation category page.
\item The lecture should be able to provide mark rollup for each category or for a specific assessment.
\item The lecture should be able to view or edit the marks of a student.
\item The lecture should be able to search for a student by providing a name, surname, or student number.
\item The lecture should be able to export marks.
\item The lecture should be able to generate a report of changes made by markers to student marks.
\end{itemize}

\subsection{Hardware Interfaces}
\label{Hardware Interfaces}
The system is intended as a mobile application for the android platform and a web server, therefore it is supported on android powered devices and also on all devices with a web browser.
\subsection{Software Interfaces}
\label{Software Interfaces}

The system is to be developed under the android operating system and via a web server using Django and Python as the server side implementation, SOAP API, and Android SDK for the mobile application.
\begin{itemize}
\item Outgoing data consist of semester marks, year marks, practical marks, assignment marks, project marks, and audit reports.
\item Incoming data consist of mark updates, assignment flags, evaluation categories, mark rollup, number of assessments, student numbers for search purposes and names and surnames.

\item communication between the user’s phone or web browser and the server will occur when:

\begin{itemize}
\item A marker adds new student marks
\item A marker updates student mark
\item A lecture updates student marks
\item A lecture specify a mark rollup for each category or each assessment
\item A student view or export their marks
\item A lecture or marker search for a student

\end{itemize}
\end{itemize}

\subsection{Communication Interfaces}
\label{Communication Interfaces}
The system will have a network server which is web based and implemented using Django, Python, and SOAP API as an interface.
The server’s purpose is to allow requests and responses between itself and the database. 
The system calls for the database system that stores students, lectures, teaching assistance, and courses information.
All the requests from the devices are sent to the SOAP API and the API will send the responses back to the devices.

\section{Technical/Non-functional requirements}
Add subsections as needed
\section{Open issues}
Add open issues here
\section{Glossary}
Add glossary text here
\end{document}
