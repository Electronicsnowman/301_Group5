\documentclass[10pt,a4paper]{article}
\usepackage[utf8]{inputenc}
\usepackage{amsmath}
\usepackage{amsfonts}
\usepackage{amssymb}
\author{Group 5}
\title{Mini project SRS}

\begin{document}
\tableofcontents
\pagebreak
\section{Introduction}
\subsection{Purpose}
The purpose of this document is to form a legally binding common ground between the stakeholders and the developers. This document will stipulate the functional as well as non-functional requirements of a system that will allow various contributors to collect, navigate and manipulate the marks of students studying at the university of Pretoria, as well as allow the students to view their marks.
\subsection{Document conventions}

\subsection{Project cope}
scope text here
\subsection{References}
reference text here
\section{System description}
description text here
\section{Functional requirements}
	\subsection{System features}
	\section*{Mark API}
	\begin{itemize}
		\item Must interface with CS website login details or is it LDAP?
		\item Must use Soap interface through out all applications.
		\item Must Facilitate marking sessions.
		\item Marking sessions must coincide with practical sessions.
		\item Marking session durations must also be personalizable by the client.
		\item Marker(s) must be able to record marks during marking sessions.
		\item Marks must be stored in a database.
		\item Database must be exportable to a .csv file.
		\item Database must have lock/unlock capabilities that are also automated depending on whether are marking session is currently happening.
		\item Unlocked databases must allow marker(s) to alter marks.
		\item Altered marks must include reason for alteration.
		\item Marker(s) must be able to alter marks from previous marking session only with the permission/knowledge of the client.
	\end{itemize}
	\subsection*{Android User Interface}
	\begin{itemize}
		\item Marker(s) must be able to select a marking session(s).
		\item Student(s) must be able to check acquired marks.
		\item Marks recorded by marker must be stored on local database.
		\item Local database must synchronize with server database.
	\end{itemize}
	\subsection*{Web User Interface}
	\begin{itemize}
		\item Must be implemented through Django.
  		\item Students must be able to view their marks
  		\item Must facilitate markers/students who don't use android enabled mobile devices.
	\end{itemize}
	\section*{Booking API}
	Must interface with Computer Science website booking system.
\subsection{Use-Cases}
\section{External interface requirements}
Add subsections as needed
\section{Technical/Non-functional requirements}
Add subsections as needed
\section{Open issues}
Add open issues here
\section{Glossary}
Add glossary text here
\end{document}